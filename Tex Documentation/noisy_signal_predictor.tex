\documentclass{beamer}
\setbeamertemplate{caption}[numbered]

\usetheme{default}
\usecolortheme{default}
\usefonttheme{serif}
\usepackage{graphicx}
\usepackage{cite}
\usepackage{amsmath}
\graphicspath{ {./images/} }


\title[Noisy Signal]{Noisy Signal Predictor}
\subtitle{with machine learning}
\author{Yusuf Muhammed Raji}
\institute[MMU]{Multimedia University}
\date[Mar 2020]{6 March, 2020}
\author{Yusuf Muhammed Raji}

\begin{document}

\begin{frame}
    \begin{titlepage}
        
    \end{titlepage}
\end{frame}

\begin{frame}{Table of Contents}
    \tableofcontents
        
\end{frame}

\AtBeginSection[]{
    \begin{frame}{Table of Contents}
        \tableofcontents[currentsection]
        
    \end{frame}
}
\section{Objective}

\begin{frame}
    \frametitle{Objective}
    \begin{itemize}
        \item The objective of is to predict filter out the noise from a
        sinusoidal signal
        \item This is similar to filtering the incoming signal using a conventional
        low-pass filter
        \item The predictor should be able to de-noise the sinusoidal signal
        with high degree of accuracy
    \end{itemize}
\end{frame}
\section{Introduction}
\begin{frame}
    \begin{itemize}{Introduction}
        \item Use of Machine learning algorithm to filter out noise from a signal
        \item s
    \end{itemize}
\end{frame}

\section{Methodology and Results}
\begin{frame}{Methodology and Results}
    The steps involved in this procedure are systematic.
    They are:
    \begin{itemize}
        \item generation of a sine wave with a uniform amplitude and single frequency
        \item adding random noise to the sinusoidal signal
        \item convert the noisy signal to a dataset
        \item select an appropriate model 
    \end{itemize}
\end{frame}
\subsection{Signal Generation}
\subsection{Noise Addition to the Signal}
\subsection{Conversion to datasets}
\subsection{Machine Learning Model}
\begin{frame}
    Regression algorithm will be used

    Providing the X value as the feature, the corresponding Y values(Amplitudes)
    can be predicted.

    To train the system, we must first provide it with many examples of data
    points, including both the predictors (in our case, the X values) and their
    labels (in our case, the corresponding Y values or amplitudes).

    
\end{frame}
\subsubsection{Model training}
\subsubsection{Model Validation}
\subsection{Performance measure}
\begin{frame}{Performance measure}
    \begin{itemize}
        \item A typical performance measure for regression problems is the Root
        Mean Square Error (RMSE), as shown in Equation \ref{eqn:rmse}
        \item It gives an idea of how much error the system typically makes in
        its predictions
        \item Higher value of RMSE means the error is large
        \begin{equation}
            RMSE \left(\mathbf{X},h\right) = \sqrt{\frac{1}{m} \sum_{i=1}^{m}\left(h\left(\mathbf{x}^{(i)}\right) - y^{(i)}\right)^{2}}
            \label{eqn:rmse}
        \end{equation}

    \end{itemize}
    
\end{frame}
\begin{frame}
    \frametitle{}
\end{frame}
\end{document}